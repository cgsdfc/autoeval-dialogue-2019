\chapter*{致谢}
\addcontentsline{toc}{chapter}{致谢}
我要感谢我的毕业设计指导老师荣文戈副教授。
在毕设一开始,我因为考研、重修课程等事情无法立刻投入工作,荣老师对此表示了极大的理解,对此我深表感谢。
在中期报告前的一个月,荣老师给了我关键的支持:他不但给了我两台学院的服务器,
而且还在紧缺的实验室工位中给我安排了一个。
在此后的毕设工作中,他不断根据我的工作汇报为我的毕设提供切实有效的指导。
没有荣老师的帮助和指导,我的毕设工作可能完全走不上正轨。
荣老师的谆谆教导和谦谦君子之风让我感受到老一辈北航人一脉相承的渊博与儒雅,能在他的指导下完成毕设是我的荣幸。

我要感谢我们计算机学院对毕设工作的高度重视和大力支持。
学院早在去年10月就召开了毕设动员大会,不久之后就组织了全学院的开题报告。
据我所知,全校范围内恐怕没有比我们学院更早的了。得益于此,我们比别的学院多出几乎半年时间做准备。
我非常感谢高小鹏院长等学院领导对我们毕设的重视,
高院长在毕设动员大会上指出了毕设过程中可能会遇到的种种挫折,告诫我们要对自己负责,要多和导师沟通等等,
这些都是我在毕设过程中一直受用的。
开题报告、中期报告和毕设答辩等环节有赖于学院的全体教职工的辛勤付出,
感谢他们不惜牺牲宝贵的休息时间来为我们这些初出茅庐的本科生提出意见。

我还想感谢密码学课程的郭华老师,她在我毕业这年教会我信息安全的重要性。
感谢G951的各位学长学姐,感谢他们在“五一”劳动节和我一起看电影,这是一段美好的回忆。
感谢学生三公寓的楼管阿姨,她总是把我们当成自己的孩子看待,热心帮助我们,关心我们。
感谢校车司机,他们一天出车十几趟,将满载的师生安全送达,还将我遗失在校车上的背包完璧归赵。
感谢在某个暴雨的凌晨,一位不知名的博士学长撑着伞送我回宿舍……即使是陌生人也带来了温暖。

最后,我要感谢我的父母:他们无私和博大的关爱,我永远铭记于心。
